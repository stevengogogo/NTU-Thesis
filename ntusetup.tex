% !TeX root = ./main.tex

% --------------------------------------------------
% 資訊設定(Information Configs)
% --------------------------------------------------

\ntusetup{
  university*   = {National Taiwan University},
  university    = {國立臺灣大學},
  college       = {電機資訊學院},
  college*      = {College of Electrical Engineering and Computer Science},
  institute     = {生醫電子與資訊學研究所},
  institute*    = {Graduate Institute of Biomedical Electronics and Bioinformatics},
  title         = {以微分方程系統解析粒線體對細胞核傳訊的系統特性},
  title*        = {Understanding the System Dynamics of\\  Mitochondrial Retrograde Signaling from a Differential Equation-based Framework},
  author        = {邱紹庭},
  author*       = {Shao-Ting Chiu},
  ID            = {R07945001},
  advisor       = {魏安祺},
  advisor*      = {An-Chi Wei},
  %date          = {2020-05-01},         % 若註解掉,則預設為當天
  %oral-date     = {2020-05-01},         % 若註解掉,則預設為當天
  DOI           = {10.5566/NTU2018XXXXX},
  keywords      = {粒線體逆訊號, 數學模型, 控制系統},
  keywords*     = {Mitochondrial retrograde signaling (RTG),mathematical modeling,control systems},
}

% --------------------------------------------------
% 加載套件(Include Packages)
% --------------------------------------------------

\usepackage[sort&compress]{natbib}      % 參考文獻
\usepackage{amsmath, amsthm, amssymb}   % 數學環境
\usepackage{ulem, CJKulem}              % 下劃線、雙下劃線與波浪紋效果
\usepackage{booktabs}                   % 改善表格設置
\usepackage{multirow}                   % 合併儲存格
\usepackage{diagbox}                    % 插入表格反斜線
\usepackage{array}                      % 調整表格高度
\usepackage{longtable}                  % 支援跨頁長表格
\usepackage{paralist}                   % 列表環境
\usepackage{subfiles}

\usepackage{lipsum}                     % 英文亂字


% --------------------------------------------------
% 套件設定(Packages Settings)
% --------------------------------------------------
