% !TeX root = ../main.tex


\begin{abstract}

中文摘要中文摘要中文摘要中文摘要中文摘要中文摘要中文摘要中文摘要中文摘要中文摘要中文摘要中文摘要中文摘要中文摘要中文摘要中文摘要中文摘要中文摘要中文摘要中文摘要中文摘要中文摘要中文摘要中文摘要中文摘要中文摘要中文摘要中文摘要中文摘要中文摘要中文摘要中文摘要中文摘要中文摘要中文摘要中文摘要中文摘要中文摘要中文摘要中文摘要中文摘要中文摘要中文摘要中文摘要中文摘要中文摘要中文摘要中文摘要中文摘要中文摘要中文摘要中文摘要中文摘要中文摘要中文摘要中文摘要中文摘要中文摘要中文摘要中文摘要中文摘要中文摘要中文摘要中文摘要中文摘要中文摘要中文摘要中文摘要中文摘要中文摘要中文摘要中文摘要中文摘要中文摘要中文摘要中文摘要中文摘要中文摘要中文摘要

\end{abstract}

\begin{abstract*}

    Mitochondrial quality control is essential to maintain cell viability. Retrograde signaling is the process in which mitochondria send quality information to regulate the nuclear genome. Moreover, mitochondria communicate with the nuclear genome via biochemical networks; the nucleus responds properly depending on the signal sent by multiple mitochondrial agents, making retrograde signaling a multiple-input-single-output (MISO) problem in communication theory. However, it is unclear how the cell extracts multidimensional mitochondrial information from dynamical concentration patterns of signaling molecules. To address this issue, we used budding yeast, \emph{Saccharomyces cerevisiae}, as a model organism to investigate the communication between the mitochondrial network and nucleus. Mitochondrial membrane potential and translocation of \pr{Rtg3}/\pr{Rtg1} are considered to be the input and output of the communication system. The mathematical model of yeast retrograde signaling was constructed by a differential equation-based framework; the parameters of the model were generated by Monte Carlo Method and fitted with the translocation data of RTG proteins. In this study, we have provided a control theory view of mitochondrial retrograde signaling that may lead to a better understanding of the intracellular communication between the mitochondrial network and the nucleus genome.

\end{abstract*}

